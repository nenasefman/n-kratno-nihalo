\documentclass[a4paper,11pt]{article}
\usepackage{amsmath}
\usepackage{amssymb}
\usepackage{graphicx}
\usepackage{geometry}
\usepackage{hyperref}
\usepackage{subcaption}


\usepackage{biblatex}
\addbibresource{literatura.bib}

\usepackage[slovene]{babel}
\usepackage[T1]{fontenc}
\usepackage[utf8]{inputenc}
\setlength{\parindent}{0pt} % lahko odstraniš, mene so sam malo motili zamiki
\setlength{\parskip}{0.5em} % -||-

\geometry{margin=2.5cm}

\title{Animacija dvojnega nihala}
\author{
UNIVERZA V LJUBLJANI, UL FMF\\
FINANČNA MATEMATIKA - 2. STOPNJA\\[1em]
Nena Šefman Hodnik, Nina Švigelj\\[1em]
Mentor: prof. dr. Andrej Bauer
}
\date{}


\begin{document}

\maketitle


\section{Osnovni primeri}

\subsection{Dvojno nihalo}

Oglejmo si primer na sliki:

\begin{figure}[h!]
    \centering
    \includegraphics[width=0.3\textwidth]{primer_dveh_slika.png}
    \caption{Dvojno nihalo \cite{scipython:doublependulum}}
    \label{fig:dvojno_vir}
\end{figure}

Imamo dve žogici z masama $m_1$ in $m_2$ na palčkah dolžine $l_1$ in $l_2$ z zanemarljivo maso.

Recimo, da je prva žogica na $(x_1, y_1)$ in druga žogica na $(x_2, y_2)$. Te koordinate izrazimo s kotoma $\theta1, \theta2$ kot
\begin{align*}
x_1 &= l_1 \sin(\theta_1), \\
x_2 &= l_1 \sin(\theta_1) + l_2 \sin(\theta_2), \\
y_1 &= -l_1 \cos(\theta_1), \\
y_2 &= -l_1 \cos(\theta_1) - l_2 \cos(\theta_2).
\end{align*}

\subsection*{Potencialna energija}

Potencialna energija je definirana kot $V = m g h$, kjer je višina žogice dana z $h = y$. \\
Za posamezni masi velja: $$h_1 = y_1, \quad h_2 = y_2.$$

Celotna potencialna energija sistema je torej
\begin{align*}
V &= V_1 + V_2 \\
&= m_1 g(-l_1 \cos\theta_1) + m_2 g(-l_1 \cos\theta_1 - l_2 \cos\theta_2).
\end{align*}

\subsection*{Kinetična energija}

Kinetična energija vsake žogice je podana z izrazom $$T = \frac{1}{2} m v^2.$$

Hitrost žogic izračunamo po formuli $v = \sqrt{\dot{x}^2 + \dot{y}^2}$.

Odvodi koordinat so: \begin{align*}
\dot{x}_1 &= l_1 \dot{\theta}_1 \cos(\theta_1), \\
\dot{x}_2 &= l_1 \dot{\theta}_1 \cos(\theta_1) + l_2 \dot{\theta}_2 \cos(\theta_2), \\
\dot{y}_1 &= l_1 \dot{\theta}_1 \sin(\theta_1), \\
\dot{y}_2 &= l_1 \dot{\theta}_1 \sin(\theta_1) + l_2 \dot{\theta}_2 \sin(\theta_2).
\end{align*}

Skupno kinetično energijo sistema torej izrazimo kot
$$T = T_1 + T_2 = \frac{m_1}{2}(\dot{x}_1^2 + \dot{y}_1^2) + \frac{m_2}{2}(\dot{x}_2^2 + \dot{y}_2^2)$$

Lotimo se sedaj zapisa sistema Euler-Lagrangeevih enačb \cite{wikipedia:lagrangian} za $\mathcal{L} = T-V$:
\begin{align*}
    \mathcal{L} =& \, \frac{m_1}{2}(\dot{x}_1^2 + \dot{y}_1^2) + \frac{m_2}{2}(\dot{x}_2^2 + \dot{y}_2^2) + m_1 g(l_1 \cos\theta_1) + m_2 g(l_1 \cos\theta_1 + l_2 \cos\theta_2)\\
    =& \, \frac{m_1}{2}l_1^2 \dot{\theta_1}^2 + \frac{m_2}{2}(l_1^2 \dot{\theta_1}^2 + l_2^2 \dot{\theta_2}^2 + 2 l_1 \dot{\theta_1} l_2 \dot{\theta_2} \cos \theta_1 \cos \theta_2 + 2 l_1 \dot{\theta_1}l_2 \dot{\theta_2} \sin \theta_1 \sin \theta_2) \\
    &+ m_1 g l_1 \cos\theta_1 + m_2 g (l_1 \cos\theta_1 + l_2 \cos\theta_2)\\
    =& \, \frac{1}{2}m_1 l_1^2 \dot{\theta_1}^2 + \frac{1}{2}m_2(l_1^2 \dot{\theta_1}^2 + l_2^2 \dot{\theta_2}^2 + 2 l_1 l_2 \dot{\theta_1} \dot{\theta_2} \cos (\theta_1 - \theta_2)) + (m_1 + m_2)l_1 g \cos \theta_1 \\
    &+ m_2 l_2 g \cos \theta_2.
\end{align*}
Želimo dobiti enačbe oblike:
$$\frac{d}{dt} \Big(\frac{\partial \mathcal{L}}{\partial \dot{\theta_i}}\Big) = \frac{\partial \mathcal{L}}{\partial \theta_i}, \quad i = 1,2.$$
Najprej izpeljimo za $\theta_1$:
\begin{align*}
    \frac{\partial \mathcal{L}}{\partial \dot{\theta_1}} =& \, m_1 l_1^2 \dot{\theta_1} + m_2 l_1^2 \dot{\theta_1} + m_2 l_1 l_2 \dot{\theta_2} \cos(\theta_1 - \theta_2),\\
    \frac{d}{dt} \Big(\frac{\partial \mathcal{L}}{\partial \dot{\theta_1}}\Big) =& \, \, m_1 l_1^2 \ddot{\theta_1} + m_2 l_1 l_2 [\ddot{\theta_2} \cos(\theta_1 - \theta_2)- \dot{\theta_2} \sin(\theta_1 - \theta_2)(\dot{\theta_1}-\dot{\theta_2})],\\
    \frac{\partial \mathcal{L}}{\partial \theta_1} =& \, -m_2 l_1 l_2 \dot{\theta_1}\dot{\theta_2} \sin (\theta_1 - \theta_2) - (m_1 + m_2) l_1 g \sin\theta_1.
\end{align*}
Torej za $i=1$ dobimo:
\begin{align*}
    \frac{d}{dt} \Big(\frac{\partial \mathcal{L}}{\partial \dot{\theta_1}}\Big) - \frac{\partial \mathcal{L}}{\partial \theta_1} =& \, (m_1 + m_2) (l_1^2 \ddot{\theta_1}) + m_2 l_1 l_2 [\ddot{\theta_2}\cos (\theta_1 - \theta_2) + \dot{\theta_2}^2 \sin(\theta_1 - \theta_2)] \\
    &+(m_1 + m_2)l_1 g \sin \theta_1 = 0 \quad /:l_1.
\end{align*}
Podobno naredimo še za $i=2$ in dobimo:
\begin{align*}
    \frac{\partial \mathcal{L}}{\partial \dot{\theta_2}} =& \, m_2 l_2^2 \dot{\theta_2} + m_2 l_1 l_2 \dot{\theta_1} \cos(\theta_1 - \theta_2),\\
    \frac{d}{dt} \Big(\frac{\partial \mathcal{L}}{\partial \dot{\theta_2}}\Big) =& \, m_2 l_2^2 \ddot{\theta_2} + m_2 l_1 l_2 [\ddot{\theta_1} \cos(\theta_1 - \theta_2) - \dot{\theta_1} \sin (\theta_1 - \theta_2)(\dot{\theta_1} - \dot{\theta_2})],\\
    \frac{\partial \mathcal{L}}{\partial \theta_2}=& \, -m_2 l_1 l_2 \dot{\theta_1} \dot{\theta_2} \sin(\theta_1 - \theta_2) - m_2 l_2 g \sin \theta_2.
\end{align*}
Iz tega zapišemo še drugo Euler-Lagrangeevo enačbo:
\begin{align*}
    \frac{d}{dt} \Big(\frac{\partial \mathcal{L}}{\partial \dot{\theta_2}}\Big) - \frac{\partial \mathcal{L}}{\partial \theta_2} =& \, m_2 l_2^2 \ddot{\theta_2} + m_2 l_1 l_2 (\ddot{\theta_1} \cos(\theta_1 - \theta_2) - \dot{\theta_1}^2 \sin (\theta_1 - \theta_2)) + m_2l_2 g \sin \theta_2 =0 \quad /:l_2
\end{align*}

Dobimo sistem diferencialnih enačb:
\begin{align*}
    \theta_1: &\quad (m_1 + m_2)[l_1 \ddot{\theta_1} + g \sin \theta_1] + m_2 l_2 [\ddot{\theta_2} \cos(\theta_1 - \theta_2) + \dot{\theta_2}^2 \sin(\theta_1-\theta_2)] = 0\\
    \theta_2: &\quad m_2 [l_2 \ddot{\theta_2} + g \sin \theta_2] + m_2 l_1 [\ddot{\theta_1} \cos(\theta_1 - \theta_2) - \dot{\theta_1}^2 \sin(\theta_1 - \theta_2)] =0
\end{align*}

\subsection{Trojno nihalo}
Kaj pa bi se zgodilo, če vzamemo trojno nihalo? Predpostavimo, da na drugo žogico pripnemo še eno vrvico dolžine $l_3$ in žogico z maso $m_3$.

Ta žogica je na položaju 
\begin{align*}
    x_3 &= l_1 \sin\theta_1 + l_2 \sin \theta_2 + l_3 \sin \theta_3,\\
    y_3 &= - l_1 \cos \theta_1 - l_2 \cos \theta_2 - l_3 \cos \theta_3,
\end{align*}
prva odvoda koordinat po času pa sta
\begin{align*}
    \dot{x_3} &= l_1 \dot{\theta_1} \cos \theta_2 + l_2 \dot{\theta_2} \cos \theta_2 + l_3 \dot{\theta_3} \cos \theta_3,\\
    \dot{y_3} &= l_1 \dot{\theta_1} \sin \theta_2 + l_2 \dot{\theta_2} \sin \theta_2 + l_3 \dot{\theta_3} \sin \theta_3.
\end{align*}

Za ta sistem imamo:
\begin{align*}
    V_3 =& \, m_1 g y_1 + m_2 g y_2 + m_3 g y_3 \\
    =& \, -(m_1 + m_2 + m_3)l_1 g \cos \theta_1 - \cos \theta_2 g l_2 (m_2 + m_3) - \cos\theta_3 l_3 m_3 g\\
    T_3 =& \, \frac{1}{2} m_1 (\dot{x_1}^2 + \dot{y_1}^2) + \frac{1}{2} m_2 (\dot{x_2}^2 + \dot{y_2}^2) + \frac{1}{2} m_3 (\dot{x_3}^2 + \dot{y_3}^2)\\
    =& \, \frac{1}{2} m_1 l_1^2 \dot{\theta_1}^2 + \frac{1}{2}m_2 [l_1^2 \dot{\theta_1}^2 + l_2 \dot{\theta_2}^2 + 2l_1l_2 \dot{\theta_1}\dot{\theta_2} \cos (\theta_1 - \theta_2)] + \frac{1}{2} m_3 [l_1^2 \dot{\theta_1}^2 + l_2 \dot{\theta_2}^2 \\
    &+ l_3^2 \dot{\theta_3}^2 + 2 l_1 l_2 \dot{\theta_1}\dot{\theta_2} \cos (\theta_1 - \theta_2) + 2 l_1 l_3 \dot{\theta_1}\dot{\theta_3} \cos (\theta_1 - \theta_3) + 2 l_2 l_3 \dot{\theta_2}\dot{\theta_3} \cos (\theta_2 - \theta_3)]\\
    \mathcal{L} =&\, T_3 - V_3
\end{align*}

Podobno kot za dvojno nihalo tudi tu dobimo sistem enačb za $\theta_i$, $i=1,2,3$. Če jih malo preuredimo, dobimo:
\begin{align*}
    \theta_1: &\quad (m_1 + m_2 + m_3) [l_1 \ddot{\theta_1} + g \sin \theta_1] + (m_2 + m_3) l_2 [\ddot{\theta_2} \cos (\theta_1 - \theta_2) + \dot{\theta_2}^2 \sin (\theta_1 - \theta_2)] \\
    & + m_3 l_3 [\ddot{\theta_3} \cos(\theta_1 - \theta_3) + \dot{\theta_3}^2 \sin (\theta_1 - \theta_3)]  = 0\\
    \theta_2: &\quad (m_2 + m_3) [l_2 \ddot{\theta_2} + g \sin \theta_2] + (m_2 + m_3) l_1 [\ddot{\theta_1} \cos (\theta_1 - \theta_2) - \dot{\theta_1}^2 \sin(\theta_1 - \theta_2)] \\
    & + m_3 l_3 [\ddot{\theta_3} \cos(\theta_2 - \theta_3) + \dot{\theta_3}^2 \sin(\theta_2 - \theta_3)] = 0\\
    \theta_3: &\quad m_3 [l_3 \ddot{\theta_3} - g \sin \theta_3] + m_3 l_1 [\ddot{\theta_1} \cos(\theta_1 -\theta_3) - \dot{\theta_1}^2 \sin (\theta_1 - \theta_3)] \\
    & + m_3 l_2 [\ddot{\theta_2}\cos(\theta_2 - \theta_3) - \dot{\theta_2}^2 \sin(\theta_2 - \theta_3)] = 0
\end{align*}


% \textit{Zaenkrat sva dobili idejo za reukrzijo, ki izgleda nekako tako?:}
% \begin{align*}
%     \text{Za} \ \theta_1: &\quad (\sum_{i=1}^n m_i)[l_1 \ddot{\theta_1} + g \sin \theta_1] + \sum_{i=2}^n (\sum_{j=i}^n m_j) l_i (\ddot{\theta_i} \cos (\theta_1 - \theta_i) + \dot{\theta_i}^2 \sin(\theta_1 - \theta_i)) = 0\\
%     \text{Za} \ \theta_n: &\quad m_n [l_n \ddot{\theta_n} + g \sin \theta_n] + m_n \sum_{i=1}^{n-1} l_i [\ddot{\theta_i} \cos(\theta_i - \theta_n) - \dot{\theta_i}^2 \sin (\theta_i - \theta_n)] = 0
% \end{align*}

\section{Postopek animacije}
Za sistem dvojnega nihala sva v programskem jeziku \texttt{python} napisali funkcijo \\ \texttt{resen\_sistem\_n\_simbolicno}, ki simbolično zapiše sistem diferencialnih enačb za izbrano število nihal $n$. Potem z uporabo funkcije \texttt{resen\_sistem\_n\_numericno} glede na
\begin{itemize}
    \item mase $m_1, m_2, \ldots, m_n,$
    \item dolžine vrvic $l_1, l_2,\ldots, l_n,$
    \item začetne pogoje $\theta_1, \dot{\theta_1}, \theta_2, \dot{\theta_2},\ldots, \theta_n, \dot{\theta_n},$
    \item  korak $dt,$
    \item  čas nihanja $t_{max}$
\end{itemize}

za vsak $t_k = k\cdot dt, \ k = 0,1,\ldots, \lfloor \frac{tmax}{dt} \rfloor$ rešimo sistem diferencialnih enačb in dobimo matriko z vrsticami $[\theta_1(t_k), \dot{\theta_1}(t_k), \theta_2(t_k), \dot{\theta_2}(t_k), \ldots, \theta_n(t_k), \dot{\theta_n}(t_k)], \quad k = 0,1,\ldots, \lfloor \frac{tmax}{dt} \rfloor$. Za numerično reševanje diferencialnih enačb sva uporabili funkcijo \texttt{solve\_ivp} z RK45 metodo. Ta metoda je podrobneje opisana v \cite{scipy:solveivp}. V nekaterih prejšnjih verzijah pa sva uporabili tudi funkcijo \texttt{odeint} (\textit{ordinary differential equation integration}), ki je opisana v \cite{scipy:odeint}.

S pomočjo dobljenih vrednosti nato izračunamo koordinate vsake žogice v vsakem trenutku in jih narišemo z uporabo funkcije \texttt{narisi\_sliko\_2} za risanje preprostega dvojnega nihala. Podobno sva naredili tudi za trojno nihalo s funkcijo \texttt{narisi\_sliko\_3}.

\begin{figure}[h!]
    \centering
    \begin{subfigure}[b]{0.45\textwidth}
        \centering
        \includegraphics[width=\textwidth, trim=15cm 1cm 10cm 10cm, clip]{primer_narisan_dvojno.png}
        \caption{Primer slike dvojnega nihala.}
        \label{fig:primer_dvojno}
    \end{subfigure}
    \begin{subfigure}[b]{0.45\textwidth}
        \centering
        \includegraphics[width=\textwidth, trim=17cm 1cm 15cm 12cm, clip]{primer_narisan_trojno.png}
        \caption{Primer slike trojnega nihala.}
        \label{fig:primer_trojno}
    \end{subfigure}
    \caption{Primeri narisanih nihal.}
\end{figure}

Dobljene slike nato s pomočjo orodja \texttt{ffmpeg} in določenega \texttt{fps} (\textit{frames per second oz. sličic na sekundo}) združimo s funkcijo \texttt{ustvari\_video} v video datoteko.

\section{Primeri risanja nihal}
Poleg osnovnih dveh nihal, ki sta prikazana na sliki \ref{fig:primer_dvojno} in \ref{fig:primer_trojno} sva se lotili še nekaj bolj zanimivih primerov. Prav tako sva se igrali z barvami, zato se bodo morda barve na slikah med seboj malo razlikovale. Velik del videoposnetkov je dostopen na \href{https://github.com/nenasefman/n-kratno-nihalo/tree/main/video}{GitHub repozitoriju}. 

\subsection{Dvojna nihala v prostoru}
V tem primeru sva narisali več dvojnih nihal, ki so obešena na različne višine v prostoru. Vsako izmed nihal ima svoje začetne pogoje, torej kota $\theta_1$ in $\theta_2$, mase in dolžini vrvic. Vsa nihala se spustijo iz mirovanja.

\begin{figure}[h!]
    \centering
    \includegraphics[width=0.6\textwidth]{primer_10x10_slikice_nihal.png}
    \caption{$10 \times 10$ Dvojno nihalo}
    \label{fig:dvojno_1010}
\end{figure}



Primer takšenga risanja je na sliki \ref{fig:dvojno_1010}.

\subsection{Dvojna nihala kot kvadratki}

Ker risanje velikega števila dvojnih nihal v mrežo kmalu postane praktično nemogoče, sva se odločili prikazovati samo še njihove barve. Te sva določali na različne načine preko kotov $\theta_1, \theta_2$ in kotnih hitrosti $\dot{\theta_1}, \dot{\theta_2}$ nihal v posameznih trenutkih (funkcije za barvanje lahko najdete na repozitoriju v datoteki \texttt{barve.py}). Na ta način sva lahko narisali mnogo mnogo večje mreže. Začetne pogoje nihal, ki so si blizu skupaj, sva nastavili tako, da so si zelo podobni, ter na ta način vizualizirali fazne portrete obravnavanih diferencialnih enačb. Primer takšnega risanja je prikazan na slikah \ref{fig:crnobeli_kvadratki} in \ref{fig:barvni_kvadratki}.

\begin{figure}[h!]
    \centering
    \includegraphics[width=0.6\textwidth]{crnobeli_kvadratki.png}
    \caption{Primer risanja dvojnih nihal predstavljenih kot barvni kvadratki.}
    \label{fig:crnobeli_kvadratki}
\end{figure}

\begin{figure}[h!]
    \centering
    \includegraphics[width=0.6\textwidth]{barvni_kvadratki.png}
    \caption{Primer risanja dvojnih nihal predstavljenih kot barvni kvadratki.}
    \label{fig:barvni_kvadratki}
\end{figure}


\subsection{Dvojna nihala v koordinatnem sistemu $\theta_1$ in $\theta_2$}


Dvojna nihala lahko narišemo tudi kot točke v koordinatnem sistemu, kjer je $x$ koordinata enaka kotu drugega nihala $\theta_2$, $y$ koordinata pa kotu prvega nihala $\theta_1$. Vsako nihalo ima svoje začetne pogoje, torej kota $\theta_1$ in $\theta_2$, mase in dolžini vrvic. Tudi tu se vsa nihala spustijo iz mirovanja. Da pa bi bila nihala bolj vidna, sva jim dodali še sledi - repe, ki za določen čas \texttt{T\_rep} hrani in riše pretekle položaje nihal. Primer risanja nihal v koordinatnem sistemu je na sliki \ref{fig:repki_primer}.

\begin{figure}[h!]
    \centering
    \includegraphics[width=0.6\textwidth, trim=15cm 5cm 13cm 5cm, clip]{crvi_primer.png}
    \caption{Primer risanja dvojnih nihal v koordinatnem sistemu z repi.}
    \label{fig:repki_primer}
\end{figure}












\pagebreak
\nocite{*}
\printbibliography


\end{document}